% hello.tex
\documentclass[UTF8,a4paper]{ctexrep}

% for hyperlinks
\usepackage{hyperref}
\usepackage{url}
% For figures
\usepackage{graphicx}
\usepackage{subfigure}
% math packages
\usepackage{amsmath}
\usepackage{amsfonts}
\usepackage{amsopn}
\usepackage{ifthen}
\usepackage{natbib}
\usepackage{enumerate}
\usepackage{booktabs}
\usepackage{fancyhdr}
\usepackage{fancyvrb}
\usepackage{float}
\usepackage{tikz}
\usetikzlibrary{mindmap,trees}
% for hyperlinks


\begin{document}
\hypersetup{CJKbookmarks=true}
%标题摘要
\title{缺陷检测项目进度}
\author{林涛 刘俊峰}
\date{2019年 6月 26}
\maketitle

%摘要
%\begin{abstract}
%\end{abstract}

%\chapter{图像分类}
%\usetikzlibrary{trees}
%\part{大健康}
%\chapter{项目开发思维导图}

\chapter{图像分类}
\section{思维导图}

\pagestyle{empty}
\begin{figure}[htb]
\centering
\begin{tikzpicture}
  \path[mindmap,concept color=black,text=white]
    node[concept] {图像分类项目}
    [clockwise from=0]

    % 节点1
    child[concept color=green!50!black] {
      node[concept] {1.数据采集}
      [clockwise from=100]
      child { node[concept] {分类类别数量:单个图像多标签} }
      child { node[concept] {数据的纯度} }
    }
    % 节点2
    child[concept color=blue!50!white] {
      node[concept] {2.数据预处理}
      [clockwise from=30] % 旋转的角度
      child { node[concept] {放缩(resize)} }
      child { node[concept] {旋转,平移} }
      child { node[concept] {归一化(正态分布,均匀分布)}}
    }
    child[concept color=red] {
      node[concept] {3.模型训练}
      [clockwise from=-60]
      child [concept color=orange]{
      node[concept] {a.模型结构}
      [clockwise from=-30]
      child { node[concept] {基础模型} }
      child { node[concept] {速度} }
      child { node[concept] {精度} }
       }
      child[concept color=orange]{
      node[concept] {b.调参}
      [clockwise from=-100]
      child { node[concept] {参数初始化}}
      child { node[concept] {学习率}}
      child { node[concept] {weight decay}}
      child { node[concept] {batchsize}}
       }
      child [concept color=orange]{
       node[concept] {c.增加泛化能力}
       [clockwise from=-180]
       child {node[concept]{优化器}}
       child {node[concept]{Regularization}}
       child {node[concept]{增加数据的质量和数量}}
       }
     }
    child[concept color=orange] {
    node[concept] {4.模型预测}
    [clockwise from=150]
    child {node[concept]{评价指标(Accuracy/confusion matrix)}}
    child {node[concept]{过拟合/欠拟合}}
    };
\end{tikzpicture}
\end{figure}


\begin{tikzpicture}[edge from parent fork right,grow=right,level distance=3cm,level 1/.style={sibling distance=4cm},
level 2/.style={sibling distance=1cm}]
\node[text width=1cm] {Idea General}
child {node {Some idea}}
child {node {Some idea}
child {node {Details}}
child {node {Details}}
child {node {Details}}
};
\end{tikzpicture}
%\chapter{章标题}
这一章我们来介绍这些内容

%\section{节标题}
这是第一节分类模型

%\subsection{小节标题}
数据采集

%\subsubsection{子节标题}
七分类数据集

%\paragraph{段标题}
数据介绍
%
%\subparagraph{子段标题}
每个类别的数据量
\end{document}
